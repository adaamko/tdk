%----------------------------------------------------------------------------
% Abstract in hungarian
%----------------------------------------------------------------------------
\chapter*{Kivonat}\addcontentsline{toc}{chapter}{Kivonat}



\vfill

%----------------------------------------------------------------------------
% Abstract in english
%----------------------------------------------------------------------------
\chapter*{Abstract}\addcontentsline{toc}{chapter}{Abstract}
%--------------------------------------------------------------------------------------
% semantic parsing, graftranszformacio altalaban
The main task of semantic parsing is to automatically build semantic representation from the input, so we can model the meaning
of raw texts. If we model meaning as directed graphs of concept and we can build them from syntax trees that represents the structure of 
sentences, then we can define the whole process as one complex graph transformation.

%--------------------------------------------------------------------------------------
% machine comprehension
Representations of natural language semantics
are rarely used explicitly in state of the art systems
for popular semantics tasks such as measuring semantic
similarity or machine comprehension. These systems mostly use word embeddings as representation of word meaning.

In this paper we use graphical representations and transformations as simple but powerful tools for recognizing entailment and we describe a method using semantic parsing system \texttt{4lang} (Recski:2016) and applying it on
the 2018 Semeval Task \textit{Machine comprehension using commonsense
knowledge}. This task 
requires participants
to train systems that can choose the correct
answer to simple multiple choice questions
based on short passages describing simple chains
of events. Data for both training and testing is extracted
from the \texttt{MCScript} dataset (Ostermann
et al., 2018). The top two systems, \texttt{HFL-RC} (Chen
et al., 2018) and \texttt{Yuanfudao}
achieved accuracy scores of $84.15\%$ and $83.95\%$
on the test data, respectively.

%--------------------------------------------------------------------------------------
% baseline
First we will present a strong baseline on this task using only semantic graphs and similarities among them, followed by 
a description of a state-of-the art system \texttt{Yuanfudao} (Wang et al., 2018) and our experiments with it where
%--------------------------------------------------------------------------------------
% rendszer, amivel foglalkoztunk
we used our baseline as an extra feature for improving the neural network. The choice of the system was obvious, because the source code is publicly
available
and it already employs successfully
a knowledge base representing semantic
relationships among pairs of words \textit{ConceptNet}.
Our results
suggest that these features achieve a .5
percentage point improvement, and the \textit{ConceptNet} could be replaced by our semantic model.
\vfill

