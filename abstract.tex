%----------------------------------------------------------------------------
% Abstract in hungarian
%----------------------------------------------------------------------------
\chapter*{Kivonat}\addcontentsline{toc}{chapter}{Kivonat}
%--------------------------------------------------------------------------------------
% semantic parsing, graftranszformacio altalaban
A szemantikai elemzés célja, hogy természetes nyelvi adathoz készíthessünk szemantikai reprezentációt, 
így tudjuk modellezni a szoveg jelentését. Ha a nyelvi jelentést fogalmak irányított gráfjaival reprezentáljuk,
ezeket pedig a mondat szintaktikai szerkezetét reprezentáló fákból kell előállítanunk, akkor a teljes feladat egyetlen komplex
gráf-transzformációként definiálható.

A népszerű szemantikai feladatokra, mint a szemantikai hasonlóság mérése, vagy a gépi szövegértés, ritkán használják a természetes
nyelv szemantikájának a reprezentációját, főleg a state-of-the art rendszerekben. Ezek a rendszerek többnyire szó embeddingeket használnak a szavak jelentésének
ábrázolására, amik a szavak jelentését legfeljebb néhány száz dimenziós valós vektorként ábrázolják.

Ebben a dolgozatban mi gráf-reprezentációkat és ezek transzformációit használjuk, mint egyszerű, ám hatékony
eszközök a következmény viszony felismerésére, valamint leírunk egy módszert a \texttt{4lang} szemantikus elemzőrendszer\cite{Recski:2016d} használatára a 2018-as \textit{Semeval Task Machine comprehension using commonsense knowledge}\footnote{\url{https://competitions.codalab.org/competitions/17184}} kapcsán.
Ez a feladat azt kívánja a résztvevőktől, hogy olyan rendszereket tanítsanak fel, amelyek ki tudják választani a megfelelő választ az egyszerűbb, több válaszlehetőséget kínáló
kérdéseknél rövid eseményleíró szövegek elolvasása után. A tanító és teszt adat az \texttt{MCSript}\cite{Ostermann:2018} adathalmaz részhalmazából lett kinyerve. A két legjobb rendszer, \texttt{HFL-RC}\cite{Chen:2018} és Yuanfudao\cite{Wang:2018} rendre 84,15\% és 83,95\% pontosságot ért el a teszt adaton.

Először bemutatunk egy hatékony baseline-t ezen a feladaton csupán a szemantikus gráfok és a köztük lévő hasonlóságok felhasználásával. Ezt követően leírjuk a Yuanfudao state-of-the art
rendszert és az ezzel végzett kísérletezéseinket, amelyek során a baseline-unkat extra feature-ként felhasználva javítottunk a rendszer pontosságán.
Ennek a kiválasztása magától értetődő volt, mivel a forráskód nyilvánosan elérhető, és már sikeresen alkalmazott tudás alapú reprezentációt a szópárok közötti szemantikai kapcsolatokra,
a \textit{ConceptNet}-et. Eredményeink azt mutatják, hogy ezzel a módosítással \texttt{0.5} százalékpont növekedés érhető el, és a \textit{ConceptNet} helyettesíthető a mi szemantikus modellünkkel. 
\vfill

%----------------------------------------------------------------------------
% Abstract in english
%----------------------------------------------------------------------------
\chapter*{Abstract}\addcontentsline{toc}{chapter}{Abstract}
%--------------------------------------------------------------------------------------
% semantic parsing, graftranszformacio altalaban
The main task of semantic parsing is to automatically build semantic representation from the input, so we can model the meaning
of raw texts. If we model meaning as directed graphs of concept and we can build them from syntax trees that represent the structure of 
sentences, then we can define the whole process as one complex graph transformation.

%--------------------------------------------------------------------------------------
% machine comprehension
Representations of natural language semantics are rarely used explicitly in state-of-the art systems for popular semantics tasks such as measuring semantic similarity or machine comprehension. These systems mostly use word embeddings as representation of word meaning.

In this paper we use graphical representations and transformations as simple but powerful tools for recognizing entailment and we describe a method using semantic parsing system \texttt{4lang} (Recski:2016) and applying it on the 2018 Semeval Task \textit{Machine comprehension using commonsense knowledge}. This task requires participants to train systems that can choose the correct answer to simple multiple choice questions based on short passages describing simple chains of events. Data for both training and testing is extracted from the \texttt{MCScript} dataset (Ostermann et al., 2018). The top two systems, \texttt{HFL-RC} (Chen et al., 2018) and \texttt{Yuanfudao} achieved accuracy scores of $84.15\%$ and $83.95\%$ on the test data, respectively.

%--------------------------------------------------------------------------------------
% baseline
First we will present a strong baseline on this task using only semantic graphs and similarities among them, followed by 
a description of a state-of-the art system \texttt{Yuanfudao} (Wang et al., 2018) and our experiments with it where
%--------------------------------------------------------------------------------------
% rendszer, amivel foglalkoztunk
we used our baseline as an extra feature for improving the neural network. The choice of the system was obvious, because the source code is publicly available and it already employs successfully a knowledge base representing semantic relationships among pairs of words, \textit{ConceptNet}.
Our results suggest that these features achieve a .5 percentage point improvement, and the \textit{ConceptNet} could be replaced by our semantic model.
\vfill

