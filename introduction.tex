\chapter{Introduction}
\label{chap:Introdu}
%----------------------------------------------------------------------------
\section{Natural Language Processing}
While computers can be easily programmed to understand structured data, such as tables and spreadsheets, it can be rather challenging for them
to understand human communication. Because there is a vast difference in the magnitude of the unstructured data compared to the structured ones, there is a high demand
for tools, that can deal with raw text. Thats where NLP comes in. It contains high variety of tools, that we have to use, when we need to deal with natural input.

Every day, we come into contact with human communication, we say a lot of words to other people, and they try to interpret them even when the context of the saying 
isn't necesseraily complete. The listeners can use their common knowledge to fill the needed information. We resolve ambiguities, misunderstanding, and can even understand words 
we have never heard before just from the context of the communication.
Even though these tasks are trivial for us, for a computer it can be really hard.

%--------------------------------------------------------------------------------------
% egyszer� multtol jelenig bevezeto
The interest in NLP research began in the 1950s, the early phase was mainly focused on MT (Machine Translation), because after the World War II, people
recognised the importance of the translation from one language to another, and hoped to do it automatically.
However MT is still a very difficult nowadays, so these researches discovered early the main challenges of the syntactic and semantic parsing.
As time passed, researches embraced new areas of NLP as more advanced technology and knowledge became available. Now that we live in a world where
computers and smartphones, collecting data became incredibly easy, as a result, statistical NLP drew attention because these models thrive off big data, but one cannot ignore 
simple rule based methods, which can also be a very powerful method, especially using them as a hybrid model with statistical methods.
%--------------------------------------------------------------------------------------
% r�vid nlp pipeline

Building NLP applications requires many levels of analysis.
The typical pipeline structured as follows:
\begin{itemize}
	\item First we need to tokenize our input text, which means breaking up the text into meaningful elements, especially into words
	\item After we tokenized our text, we need word analysis called \texttt{morphology}, which is concerned with the structure of words.
	\item Part of speech explains how a word is used in a sentence, and tagging is a process marking words to its particular part of speech.
	\item The main task of syntactic parsing is to analyze the grammatical structure of a sentence. Given a set of words, a parser forms units (subjects, verbs, etc..) according to some grammatical formalism.
	There are two main type of syntactic parsers:
	\begin{itemize}
		\item Consituency parsers produce trees, that represents the grammatical structure.
		\item Dependency parsers are the more popular nowadays. They represent the structure of a sentence as a dependency tree.
	\end{itemize}
	\item At this point we have various ways to analyze a text, but without modeling its \texttt{meaning}. Semantic is the study of meaning, and semantic parsing is a task to find a representation and assign it to the text. This task will be the main topic of my work.
\end{itemize}